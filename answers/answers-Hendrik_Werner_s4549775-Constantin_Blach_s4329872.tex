\documentclass[12pt]{article}

\usepackage{enumerate}
\usepackage{amsmath}

\author{
	Hendrik Werner s4549775
	\and Constantin Blach s4329872
}

\title{Assignment 1}

\begin{document}
\maketitle

\section*{1.1}
The vectors $x, y, z, w, s$ are defined on the exercise sheet.

\subsection*{1.1.1}
\begin{enumerate}[a]
	\item %a
	$v = 3x + y =
	\begin{pmatrix}
		21\\ 28\\ 35\\ 42\\ 49\\ 56\\ 63
	\end{pmatrix}$

	\item %b
	$x \cdotp y = 1057$

	\item %c
	\item %d
	$z - 1 =
	\begin{pmatrix}
		-.3\\ 0\\ .3\\ .6\\ .9\\ 1.2\\ 1.5\\ 1.8
	\end{pmatrix}$

	\item %e
	We used a loop and Python's feature that you can access a list form the end with negative indexes. After replacing the last 3 values with $4$ we get:

	$x =
	\begin{pmatrix}
		6\\ 7\\ 8\\ 9\\ 4\\ 4\\ 4
	\end{pmatrix}$

	\item %f
	$r = 2w - 5 =
	\begin{pmatrix}
		-3\\ -3\\ -5\\ -4\\ -3\\ -2\\ -1\\ -5\\ -5
	\end{pmatrix}$
\end{enumerate}

\subsection*{1.1.2}
The matrices $M, N, P$ are defined on the exercise sheet.

\begin{enumerate}[a]
	\item %a
	$A = MN + N =
	\begin{pmatrix}
		37 & 19\\
		107 & 59\\
		103 & 56
	\end{pmatrix}$

	\item %b
	$B = N^T * M =
	\begin{pmatrix}
		76 & 99 & 65\\
		24 & 35 & 31
	\end{pmatrix}$

	\item %c
	\item %d
	\item %e
\end{enumerate}

\section*{1.2}
\subsection*{1.2.1}
\begin{enumerate}[a]
	\item %a
	\item %b
\end{enumerate}

\subsection*{1.2.2}
\begin{enumerate}[a]
	\item %a
	\item %b
	\item %c
	\item %d
	\item %e
\end{enumerate}

\end{document}
